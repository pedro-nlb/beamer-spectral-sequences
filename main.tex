\documentclass[notheorems, hyperref={backref}]{beamer}

%% KEY LINES IN THIS TEX FILE %% (enter line number+gg to go)

%
% LOCAL FONT DEFINITIONS -- need to come first
%
%\usepackage{mathpazo}
\usepackage{libertine}
\usepackage[libertine]{newtxmath}
\usefonttheme[onlymath]{serif}
\usepackage{animate}

%
% STANDARD PREAMBLE
%
\input{preamble}
\allowdisplaybreaks

%
% ABOUT FONT DEFINTIONS IN THE PREAMBLE
%
% Mathscr for sheaves use \sA, where A can be any letter. Exceptions and additions:
% % \E (vector bundles)
% % \F (coherent sheaves)
% % \G (coherent sheaves)
% % \hom (sheaf hom)
% % \I (ideal sheaves)
% % \L (line bundles)
% % \M (line bundles)
% % \O (structure sheaf)
% % \w (canonical sheaf)
%
% Mathcal use \calA. Exceptions and additions:
% % \U (open cover)
% % \X (families of varieties)
% % \Y (families of varieties)
%
% Mathbb use \bbA. Exceptions and additions:
% % \A (affine space)
% % \C (complex numbers)
% % \Gm (puctured affine line)
% % \k (field)
% % \N (natural numbers)
% % \P (projective space)
% % \Q (rational numbers)
% % \R (real numbers)
% % \V (geometric vector bundle)
% % \Z (integers)
%
% Boldfont for categories use \bfA. Additions:
% % \Cat (categories)
% % \Coh (coherent sheaves)
% % \D (derived category)
% % \Db (bounded derived category)
% % \K (homotopy category)
% % \Mod (modules)
% % \PSh (presheaves)
% % \QCoh (quasi-coherent sheaves)
% % \Set (sets)
% % \Sh (sheaves)
% % \Top (topological spaces)
% % \Vec (vector bundles)
%
% Mathfrak for ideals
% % From \a to \e
% % \m and \n for maximal ideals

%
% THEOREM ENVIRONMENTS
%
% Theorems, propositions, etc (dark green)
\theoremstyle{darkgreentheorem}
\newtheorem{thm}{Theorem}
\newtheorem{lm}[thm]{Lemma}
\newtheorem{prop}[thm]{Proposition}
\newtheorem{cor}[thm]{Corollary}
\newtheorem{conj}[thm]{Conjecture}
% Definitions (dark blue)
\theoremstyle{darkbluedefinition}
\newtheorem{defn}[thm]{Definition}
% Examples (dark red)
\theoremstyle{darkredexample}
\newtheorem{exa}[thm]{Example}
% Remarks (black)
\theoremstyle{remark}
\newtheorem{rem}[thm]{Remark}
\newtheorem{nota}[thm]{Notation}
\newtheorem{fact}[thm]{Fact}
\newtheorem{q}[thm]{Question}
\newtheorem{pbl}[thm]{Problem}

%
% THEOREM CROSS-REFERENCING
%
\crefname{thm}{theorem}{theorems}
\Crefname{thm}{Theorem}{Theorems}
\crefname{lm}{lemma}{lemmas}
\Crefname{lm}{Lemma}{Lemmas}
\crefname{prop}{proposition}{propositions}
\Crefname{prop}{Proposition}{Propositions}
\crefname{cor}{corollary}{corollaries}
\Crefname{cor}{Corollary}{Corollaries}
\crefname{conj}{conjecture}{conjectures}
\Crefname{conj}{Conjecture}{Conjectures}
\crefname{defn}{definition}{definitions}
\Crefname{defn}{Definition}{Definitions}
\crefname{exa}{example}{examples}
\Crefname{exa}{Example}{Examples}
\crefname{rem}{remark}{remarks}
\Crefname{rem}{Remark}{Remarks}
\crefname{nota}{notation}{notations}
\Crefname{nota}{Notation}{Notations}
\crefname{fact}{fact}{facts}
\Crefname{fact}{Fact}{Facts}
\crefname{q}{question}{questions}
\Crefname{q}{Question}{Questions}
\crefname{pbl}{problem}{problems}
\Crefname{pbl}{Problem}{Problems}

%
% MATH OPERATORS
%
\DeclareMathOperator{\Hom}{Hom}
\DeclareMathOperator{\gr}{gr}
\DeclareMathOperator{\im}{im}

%
% OTHER COMMANDS
%
\newcommand{\ot}{\otimes}
\newcommand{\op}{\oplus}
\newcommand{\filt}{F^{\bullet}}

%
% TITLE PAGE INFORMATION
%
\title[Basic Notions --- Spectral Sequences]{Basic Notions --- Spectral Sequences}
\author{Pedro Núñez}
\institute{University of Freiburg}
\date{10th December 2020}
 
%
% LINKS AND PDF OPTIONS
%
\makeatletter
\hypersetup{
  %pdfauthor={\authors},
  pdftitle={\@title},
  %pdfsubject={\@subjclass},
  %pdfkeywords={\@keywords},
  %pdfstartview={Fit},
  %pdfpagelayout={TwoColumnRight},
  %pdfpagemode={UseOutlines},
  bookmarks,
  colorlinks,
  linkcolor=linkblue,
  citecolor=linkblue,
  urlcolor=linkblue}
\makeatother
\usecolortheme{orchid}
 
\begin{document}
 
\frame{\titlepage}

\begin{frame}
    \frametitle{\centerline{``You could have invented spectral sequences''\footnote{Title of the expository article \cite{cho06}}}}
    \begin{figure}
	\centering
	\includegraphics[scale=.6]{pictures/good}
    \end{figure}
\end{frame}

\begin{frame}
    \frametitle{\centerline{Notation gets ugly real quick}}
    \begin{figure}
	\centering
	\includegraphics[scale=.307]{pictures/bad}
    \end{figure}
\end{frame}

\begin{frame}
    \frametitle{Notation usually following the Stacks project \cite{stacks}}
	\begin{itemize}
	    \item Only finite dimensional vector spaces over $\Q$ for simplicity.
		\pause
	    \item (Cochain) complexes denoted with capital letters $C$ and not $C^{\bullet}$.
		\pause
	    \item Cohomology again denoted just $H(C)$ and not $H^{\bullet}(C)$.
		\pause
	    \item Cohomology is regarded as a copmlex with $0$ differentials
		\[ H(C)=\bigoplus_{i\in \Z}H^{i}(C)[-i]:=\left(\cdots H^{-1}(C)\xrightarrow{0} H^{0}(C)\xrightarrow{0}H^{1}(C) \cdots \right). \]
		\pause
	    \item If it is clear to which $C$ we refer, denote $H(C)$ simply by $H$.
		\pause
	    \item (Decreasing) filtrations are always finite for simplicity:
		\[ F^{\bullet}A \colon \quad 0=F^{n}A\subseteq \ldots \subseteq F^{0}A=A. \]
		\pause
	    \item The graded object associated to a filtration $F^{\bullet}A$ is denoted
		\[ \gr{A}=\bigoplus_{p\in \N} \gr^{p}A:=\bigoplus_{p\in \N}\left( F^{p}A/F^{p+1}A \right). \]
	\end{itemize}
\end{frame}

\begin{frame}
    \frametitle{Goal: compute cohomology with the help of a filtration}
    % Define block styles
\tikzstyle{decision} = [diamond, draw, fill=red!20, 
    text width=2.5em, text badly centered, node distance=3cm, inner sep=0pt]
\tikzstyle{start} = [draw, rectangle, fill=blue!20, 
    text width=3em, text centered, rounded corners=3mm, minimum height=2em]
\tikzstyle{block} = [rectangle, draw, fill=yellow!20, 
    text width=4em, text centered, minimum height=2em]
\tikzstyle{result} = [draw, rectangle, fill=green!20, 
    text width=11em, text centered, rounded corners=3mm, minimum height=3em]
\tikzstyle{line} = [draw, -latex']
\tikzstyle{cloud} = [draw, trapezium, trapezium left angle=75, trapezium right angle=105, fill=red!20, node distance=3cm,
    minimum height=2em]
\begin{center}
\begin{tikzpicture}[node distance = 2cm, auto]
    % Place nodes
    \node [start] (complex) {$C$};
    \node [cloud, below of=complex, node distance=20mm] (filtration) {$F^{\bullet}C$};
    \node [block, left of=filtration, node distance=3cm] (gradedC) {$\gr^{p}C$};
    \node [block, right of=filtration, node distance=3cm] (induced) {$F^{\bullet}H$};
    \node [block, below of=gradedC, node distance=23mm] (cohomGradedC) {$H(\gr^{p}C)$};
    \node [block, below of=induced, node distance=23mm] (gradedH) {$\gr^{p}H$};
    \node [decision, below of=filtration, node distance=23mm] (compare) {$\cong $?};
    \node [result, below of=compare, node distance=20mm] {$H\cong \gr H \cong \bigoplus H(\gr^{p}C)$};
\end{tikzpicture}
\end{center}
\end{frame}

\begin{frame}
    \frametitle{Starting point: a filtration on our complex}
    \begin{itemize}
	    \vspace{-5mm}
	\item If we do not have any extra information about $C^{\bullet}$, there is not much more we can do besides applying the definition
	    \[ H^{\bullet}=\ker(d^{0})[0]\op\frac{\ker(d^{1})}{\im(d^{0})}[-1]\op \frac{C^{2}}{\im(d^{1})}[-2]. \]
	    \pause
	    \vspace{-5mm}
	\item But $C^{\bullet}$ has often some \underline{useful} filtration $F^{\bullet}C^{\bullet}$, say
	    \vspace{3mm}
	    \begin{center}
		\hspace{-5mm}
		\begin{tikzcd}[ampersand replacement=\&]
		    C^{\bullet}=F^{0}C^{\bullet} \&[-20pt] \cdots 0\arrow{r} \&[-15pt] C^{0}\arrow{r}{d^{0}} \arrow{r} \&[-10pt] C^{1}\arrow{r}{d^{1}} \&[-10pt] C^{2} \arrow{r} \&[-15pt] 0\cdots \\
		    F^{1}C^{\bullet}\arrow[draw=none]{u}[sloped,auto=false]{\subseteq} \& \cdots 0\arrow{r} \& F^{1}C^{0}\arrow[draw=none]{u}[sloped,auto=false]{\subseteq}\arrow{r} \& F^{1}C^{1}\arrow[draw=none]{u}[sloped,auto=false]{\subseteq}\arrow{r} \& F^{1}C^{2}\arrow[draw=none]{u}[sloped,auto=false]{\subseteq}\arrow{r} \& 0 \cdots \\
		    0=F^{2}C^{\bullet}\arrow[draw=none]{u}[sloped,auto=false]{\subseteq} \& \cdots 0 \arrow{r} \& 0\arrow[draw=none]{u}[sloped,auto=false]{\subseteq}\arrow{r} \& 0\arrow[draw=none]{u}[sloped,auto=false]{\subseteq}\arrow{r} \& 0\arrow[draw=none]{u}[sloped,auto=false]{\subseteq}\arrow{r} \& 0 \cdots
		\end{tikzcd}
	    \end{center}
    \end{itemize}
\end{frame}

\begin{frame}
    \frametitle{How could a filtration on $C^{\bullet}$ \underline{potentially} help?}
    \pause
    \begin{enumerate}
	\item $0\subseteq F^{1}C^{\bullet}\subseteq C^{\bullet}$ induces a filtration $0\subseteq F^{1}H^{\bullet}\subseteq H^{\bullet}$ given by
	    \[ F^{1}H^{i}:=\im\left(H^{i}(F^{1}C^{\bullet})\to H^{i}(C^{\bullet})\right) \text{ for all }i\in \{0,1,2\}. \]
	    \vspace{-5mm}
	    \pause
	\item We can then recover $H^{i}$ as the associated graded vector space
	    \[ \gr(F^{\bullet}H^{i})=F^{1}H^{i}\op \left(H^{i}/F^{1}H^{i}\right)\cong H^{i} \text{ for all }i\in \{0,1,2\}. \]
	    \vspace{-5mm}
	    \pause
	\item $F^{\bullet}C^{\bullet}$ being a filtered complex means that $F^{1}C^{\bullet}\subseteq C^{\bullet}$ is a subcomplex, so that $F^{1}C^{\bullet}$ and $C^{\bullet}/F^{1}C^{\bullet}$ are complexes as well.
    \end{enumerate}
    \pause
    \vspace{5mm}
    \begin{itemize}
	\item \textbf{Q1:} Can we compute the cohomologies of $F^{1}C^{\bullet}$ and $C^{\bullet}/F^{1}C^{\bullet}$?
	    \vspace{2mm}
	    \pause
	\item \textbf{Q2:} Are these cohomologies related to $F^{1}H^{\bullet}$ and $H^{\bullet}/F^{1}H^{\bullet}$?
    \end{itemize}
\end{frame}

\begin{frame}
    \frametitle{Possible scenario I: the good}
    \begin{itemize}
	\item Suppose $C^{\bullet}=C_{1}^{\bullet}\op C_{2}^{\bullet}$ with $H^{\bullet}(C_{1}^{\bullet})$ and $H^{\bullet}(C_{2}^{\bullet})$ computable.
	    \pause
	\item Then we are done, because $H^{\bullet}(C^{\bullet})=H^{\bullet}(C_{1}^{\bullet})\op H^{\bullet}(C_{2}^{\bullet})$.
	    \pause
	\item To rephrase this with filtrations we take
	    \vspace{-3mm}
	    \[ F^{1}C^{\bullet}:=C_{1}^{\bullet} \quad \left(\quad \Rightarrow \quad C^{\bullet}/F^{1}C^{\bullet}=C_{2}^{\bullet}. \quad \right) \]
	    \vspace{-8mm}
	    \pause
	\item Then we have
	    \[ F^{1}H^{\bullet}:=\im(H^{\bullet}(F^{1}C^{\bullet})\to H^{\bullet}(C^{\bullet}))=H^{\bullet}(C_{1}^{\bullet})=H^{\bullet}(F^{1}C^{\bullet}) \]
	    and similarly $H^{\bullet}/F^{1}H^{\bullet}=H^{\bullet}(C_{2}^{\bullet})=H^{\bullet}(C^{\bullet}/F^{1}C^{\bullet})$.
	    \pause
    \end{itemize}
    \vspace{2mm}
    So in this case we can carry out our naive strategy of computing each graded piece of the induced filtration in cohomology separately:
    \hspace{-10mm}
    \[ \boxed{\underset{\text{on complex}}{\text{filtration}}}\rightsquigarrow \boxed{\underset{\text{associated to filtration}}{\text{graded pieces}}} \rightsquigarrow \boxed{\underset{\text{cohomologies}}{\text{compute}}} \rightsquigarrow \boxed{\underset{{\text{graded pieces}}}{\text{put together}}} \]
\end{frame}

\begin{frame}
    \frametitle{Possible scenario II: the bad}
    Suppose we know nothing about $C^{\bullet}$.
	The \textit{stupid filtration}
	\[
	\begin{cases}
	    F^{1}C^{\bullet}:= & \cdots \to 0\to 0 \to C^{1}\xrightarrow{d^{1}} C^{2} \to 0 \to \cdots \\
	    F^{2}C^{\bullet}:= & \cdots \to 0 \to 0 \to 0 \to C^{2}\to 0 \to \cdots 
	\end{cases}
	\]
	induces the cohomology filtration
	\[
	\begin{cases}
	    F^{1}H^{\bullet}:=\im(H^{\bullet}(F^{1}C^{\bullet})\to H^{\bullet}(C^{\bullet}))=H^{1}(C^{\bullet})[-1]\op H^{2}(C^{\bullet})[-2] \\
	    F^{2}H^{\bullet}:=\im(H^{\bullet}(F^{2}C^{\bullet})\to H^{\bullet}(C^{\bullet}))=H^{2}(C^{\bullet})[-2]
	\end{cases}
	\]
	whose graded pieces are $H^{0}(C^{\bullet})[0]$, $H^{1}(C^{\bullet})[-1]$ and $H^{2}(C^{\bullet})[-2]$.

	So in this case computing $\gr^{i}H^{\bullet}(C^{\bullet})$ is as hard as computing $H^{i}(C^{\bullet})$ and we just wasted our time with this detour.
\end{frame}

\begin{frame}
    \frametitle{Possible scenario III: the ugly}
    Suppose now that we are somewhere in between the good and the bad scenarios.
    Namely, assume that we have some filtration $0\subsetneq F^{1}C^{\bullet} \subsetneq C^{\bullet}$ which does not correspond to a direct sum decomposition but such that we can compute the graded pieces $\gr^{i}(H^{\bullet}(C^{\bullet}))$ of the filtration induced in cohomology.
\end{frame}

\begin{frame}
    \frametitle{References}
    \bibliographystyle{alpha}
    \bibliography{main.bib}
\end{frame}
 
\end{document}
