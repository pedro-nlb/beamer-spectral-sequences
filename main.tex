\documentclass[notheorems, hyperref={backref}]{beamer}

%% KEY LINES IN THIS TEX FILE %% (enter line number+gg to go)

%
% LOCAL FONT DEFINITIONS -- need to come first
%
%\usepackage{mathpazo}
\usepackage{libertine}
\usepackage[libertine]{newtxmath}
\usefonttheme[onlymath]{serif}
\usepackage{adjustbox}

%
% STANDARD PREAMBLE
%
\input{preamble}
\allowdisplaybreaks

%
% ABOUT FONT DEFINTIONS IN THE PREAMBLE
%
% Mathscr for sheaves use \sA, where A can be any letter. Exceptions and additions:
% % \E (vector bundles)
% % \F (coherent sheaves)
% % \G (coherent sheaves)
% % \hom (sheaf hom)
% % \I (ideal sheaves)
% % \L (line bundles)
% % \M (line bundles)
% % \O (structure sheaf)
% % \w (canonical sheaf)
%
% Mathcal use \calA. Exceptions and additions:
% % \U (open cover)
% % \X (families of varieties)
% % \Y (families of varieties)
%
% Mathbb use \bbA. Exceptions and additions:
% % \A (affine space)
% % \C (complex numbers)
% % \Gm (puctured affine line)
% % \k (field)
% % \N (natural numbers)
% % \P (projective space)
% % \Q (rational numbers)
% % \R (real numbers)
% % \V (geometric vector bundle)
% % \Z (integers)
%
% Boldfont for categories use \bfA. Additions:
% % \Cat (categories)
% % \Coh (coherent sheaves)
% % \D (derived category)
% % \Db (bounded derived category)
% % \K (homotopy category)
% % \Mod (modules)
% % \PSh (presheaves)
% % \QCoh (quasi-coherent sheaves)
% % \Set (sets)
% % \Sh (sheaves)
% % \Top (topological spaces)
% % \Vec (vector bundles)
%
% Mathfrak for ideals
% % From \a to \e
% % \m and \n for maximal ideals

%
% THEOREM ENVIRONMENTS
%
% Theorems, propositions, etc (dark green)
\theoremstyle{darkgreentheorem}
\newtheorem{thm}{Theorem}
\newtheorem{lm}[thm]{Lemma}
\newtheorem{prop}[thm]{Proposition}
\newtheorem{cor}[thm]{Corollary}
\newtheorem{conj}[thm]{Conjecture}
% Definitions (dark blue)
\theoremstyle{darkbluedefinition}
\newtheorem{defn}[thm]{Definition}
% Examples (dark red)
\theoremstyle{darkredexample}
\newtheorem{exa}[thm]{Example}
% Remarks (black)
\theoremstyle{remark}
\newtheorem{rem}[thm]{Remark}
\newtheorem{nota}[thm]{Notation}
\newtheorem{fact}[thm]{Fact}
\newtheorem{q}[thm]{Question}
\newtheorem{pbl}[thm]{Problem}

%
% THEOREM CROSS-REFERENCING
%
\crefname{thm}{theorem}{theorems}
\Crefname{thm}{Theorem}{Theorems}
\crefname{lm}{lemma}{lemmas}
\Crefname{lm}{Lemma}{Lemmas}
\crefname{prop}{proposition}{propositions}
\Crefname{prop}{Proposition}{Propositions}
\crefname{cor}{corollary}{corollaries}
\Crefname{cor}{Corollary}{Corollaries}
\crefname{conj}{conjecture}{conjectures}
\Crefname{conj}{Conjecture}{Conjectures}
\crefname{defn}{definition}{definitions}
\Crefname{defn}{Definition}{Definitions}
\crefname{exa}{example}{examples}
\Crefname{exa}{Example}{Examples}
\crefname{rem}{remark}{remarks}
\Crefname{rem}{Remark}{Remarks}
\crefname{nota}{notation}{notations}
\Crefname{nota}{Notation}{Notations}
\crefname{fact}{fact}{facts}
\Crefname{fact}{Fact}{Facts}
\crefname{q}{question}{questions}
\Crefname{q}{Question}{Questions}
\crefname{pbl}{problem}{problems}
\Crefname{pbl}{Problem}{Problems}

%
% MATH OPERATORS
%
\DeclareMathOperator{\Hom}{Hom}
\DeclareMathOperator{\gr}{gr}
\DeclareMathOperator{\im}{im}

%
% OTHER COMMANDS
%
\newcommand{\ot}{\otimes}
\newcommand{\op}{\oplus}
\newcommand{\filt}{F^{\bullet}}

%
% TITLE PAGE INFORMATION
%
\title[Basic Notions --- Spectral Sequences]{Basic Notions --- Spectral Sequences}
\author{Pedro Núñez}
\institute{University of Freiburg}
\date{10th December 2020}
 
%
% LINKS AND PDF OPTIONS
%
\makeatletter
\hypersetup{
  %pdfauthor={\authors},
  pdftitle={\@title},
  %pdfsubject={\@subjclass},
  %pdfkeywords={\@keywords},
  %pdfstartview={Fit},
  %pdfpagelayout={TwoColumnRight},
  %pdfpagemode={UseOutlines},
  bookmarks,
  colorlinks,
  linkcolor=linkblue,
  citecolor=linkblue,
  urlcolor=linkblue}
\makeatother
\usecolortheme{orchid}
 
\begin{document}
 
\frame{\titlepage}

\begin{frame}
    \frametitle{\centerline{``You could have invented spectral sequences''\footnote{Title of the expository article \cite{cho06}}}}
    \begin{figure}
	\centering
	\includegraphics[scale=.6]{pictures/good}
    \end{figure}
\end{frame}

\begin{frame}
    \frametitle{\centerline{Notation gets ugly very soon}}
    \begin{figure}
	\centering
	\includegraphics[scale=.307]{pictures/bad}
    \end{figure}
\end{frame}

\begin{frame}
    \frametitle{Notation and conventions \cite{stacks}}
	\begin{itemize}
	    \item Only finite dimensional vector spaces over $\Q$ for simplicity.
		\pause
	    \item (Cochain) complexes denoted with capital letters $C$ and not $C^{\bullet}$.
		\pause
	    \item Cohomology again denoted just $H(C)$ and not $H^{\bullet}(C)$.
		\pause
	    \item Cohomology is regarded as a copmlex with $0$ differentials
		\[ H(C)=\bigoplus_{i\in \Z}H^{i}(C)[-i]:=\left(\cdots H^{-1}(C)\xrightarrow{0} H^{0}(C)\xrightarrow{0}H^{1}(C) \cdots \right). \]
		\pause
	    \item If it is clear to which $C$ we refer, denote $H(C)$ simply by $H$.
		\pause
	    \item (Decreasing) filtrations are always finite for simplicity:
		\[ F^{\bullet}A \colon \quad 0=F^{n}A\subseteq \ldots \subseteq F^{0}A=A. \]
		\pause
	    \item The graded object associated to a filtration $F^{\bullet}A$ is denoted
		\[ \gr{A}=\bigoplus_{p\in \N} \gr^{p}A:=\bigoplus_{p\in \N}\left( F^{p}A/F^{p+1}A \right). \]
	\end{itemize}
\end{frame}

\begin{frame}
    \frametitle{\textsc{Goal}: compute cohomology with the help of a filtration}
    % Define block styles
\tikzstyle{decision} = [diamond, draw, fill=red!20, 
    text width=2.5em, text badly centered, node distance=3cm, inner sep=0pt]
\tikzstyle{start} = [draw, rectangle, fill=blue!20, 
    text width=3em, text centered, rounded corners=3mm, minimum height=2em]
\tikzstyle{tbc} = [draw, ellipse, fill=blue!20, 
    text width=3em, text centered, minimum height=2em]
\tikzstyle{block} = [rectangle, draw, fill=yellow!20, 
    text width=4em, text centered, minimum height=2em]
\tikzstyle{result} = [draw, rectangle, fill=green!20, 
    text width=11em, text centered, rounded corners=3mm, minimum height=3em]
\tikzstyle{line} = [draw, -latex']
\tikzstyle{cloud} = [draw, trapezium, trapezium left angle=75, trapezium right angle=105, fill=red!20, node distance=3cm,
    minimum height=2em]
\begin{center}
\begin{tikzpicture}[node distance = 2cm, auto]
    % Place nodes
    \node [start] (complex) {$C$};
    \node [cloud, below of=complex, node distance=15mm] (filtration) {$F^{\bullet}C$};
    \node [block, left of=filtration, node distance=3cm] (gradedC) {$\gr^{p}C$};
    \node [block, right of=filtration, node distance=3cm] (induced) {$F^{\bullet}H$};
    \node [block, below of=gradedC, node distance=20mm] (cohomGradedC) {$H(\gr^{p}C)$};
    \node [block, below of=induced, node distance=20mm] (gradedH) {$\gr^{p}H$};
    \node [decision, below of=filtration, node distance=30mm] (compare) {$\cong $?};
    \node [start, below of=cohomGradedC, node distance=18mm] (whatNow) {TBC};
    \node [result, below of=compare, node distance=20mm] (cohom) {$H\cong \gr H \cong \bigoplus H(\gr^{p}C)$};
    % Draw edges
    \path [line] (complex) -- (filtration);
    \path [line] (filtration) -- (gradedC);
    \path [line] (filtration) -- (induced);
    \path [line] (gradedC) -- (cohomGradedC);
    \path [line] (induced) -- (gradedH);
    \path [line] (cohomGradedC) -| (compare);
    \path [line] (gradedH) |- (compare);
    \path [line] (compare) -- node {yes}(cohom);
    \path [line, dashed] (compare) -| node [near start] {no}(whatNow);
\end{tikzpicture}
\end{center}
\end{frame}

\begin{frame}
    \frametitle{Filtration induced in cohomology}
    A filtration on the complx $F^{\bullet}C$ induces a filtration in cohomology
    \[ F^{p}H:=\im(H(F^{p}C)\to H(C))\subseteq H(C)=H, \]
    where $H(F^{p}C)\to H(C)$ is induced by the inclusion map $F^{p}C\hookrightarrow C$.
\end{frame}

\begin{frame}
    \frametitle{\textsc{Baby Example}: 2 step filtration on a length 1 complex}
    \adjustbox{scale=1.2,center}{%
	\begin{tikzcd}[ampersand replacement=\&]
	    C \&[-20pt] \&[-15pt] \&[-15pt] \&[10pt] \&[-15pt] \&[-15pt] \\[-15pt]
	    F^{0}C\arrow[draw=none]{u}[sloped,auto=false]{=} \& \cdots\arrow{r} \& 0\arrow{r} \& C^{0}\arrow{r}{d^{0}} \arrow{r} \& C^{1}\arrow{r} \& 0\arrow{r} \& \cdots \\
	    F^{1}C\arrow[draw=none]{u}[sloped,auto=false]{\subseteq} \& \cdots\arrow{r} \& 0\arrow{r} \& F^{1}C^{0}\arrow[draw=none]{u}[sloped,auto=false]{\subseteq}\arrow{r}{d^{0}|_{F^{1}C^{0}}} \& F^{1}C^{1}\arrow[draw=none]{u}[sloped,auto=false]{\subseteq}\arrow{r} \& 0 \arrow{r} \&[-15pt]\cdots \\
	    F^{2}C\arrow[draw=none]{u}[sloped,auto=false]{\subseteq} \& \cdots\arrow{r} \& 0 \arrow{r} \& 0\arrow[draw=none]{u}[sloped,auto=false]{\subseteq}\arrow{r} \& 0\arrow[draw=none]{u}[sloped,auto=false]{\subseteq}\arrow{r}\& 0 \arrow{r} \&[-15pt]\cdots \\[-15pt]
	    0\arrow[draw=none]{u}[sloped,auto=false]{=} \& \& \& \& \& \&
	\end{tikzcd}
    }
\end{frame}

\begin{frame}
    \frametitle{\textsc{Baby Example}: filtration induced in cohomology}
    \vspace{-3mm}
    We first look at the map induced by $F^{1}C\hookrightarrow C$ in cohomology:

    \vspace{2mm}
    \adjustbox{scale=1.1,center}{%
	\begin{tikzcd}[ampersand replacement=\&]
	    \hspace{5mm}H(F^{1}C)\mbox{ }\mbox{ }\mbox{ }=\arrow{d} \&[-20pt] \hspace{2mm}\ker\left(d^{0}|_{F^{1}C}\right)[0]\op \left(\frac{F^{1}C^{1}}{\im\left(d^{0}|_{F^{1}C}\right)}\right)[-1]\arrow{d} \\
	    \hspace{7mm}H(C)\quad = \& \hspace{3.5mm}\ker(d^{0})[0]\op \left(\frac{C^{1}}{\im(d^{0})}\right)[-1]
	\end{tikzcd}
    }
    \pause
    
    \vspace{1cm}
    The filtration on cohomology was given by its image, hence
    \vspace{3mm}

    \adjustbox{scale=1.2,center}{%
    \begin{tcolorbox}[colback=yellow!25!white,colframe=orange!5!white,text width=8cm]
    \vspace{-3mm}
    \[ F^{1}H=\ker\left(d^{0}|_{F^{1}C}\right)[0]\op \left(\frac{F^{1}C^{1}+\im(d^{0})}{\im(d^{0})}\right)[-1] \]
    \end{tcolorbox}}
\end{frame}

\begin{frame}
    \frametitle{\textsc{Baby Example}: graded cohomology pieces}
    Recall that $\gr^{0}H:=F^{0}H/F^{1}H=H/F^{1}H$, hence (after $\cong$-theorem)
    \vspace{3mm}

    \adjustbox{scale=1.2,center}{%
    \begin{tcolorbox}[colback=yellow!25!white,colframe=orange!5!white,text width=8cm]
    \vspace{-3mm}
	\[ \gr^{0}H=\left(\frac{\ker(d^{0})}{\ker\left(d^{0}|_{F^{1}C}\right)}\right)[0]\op \left(\frac{C^{1}}{F^{1}C^{1}+\im(d^{0})}\right)[-1] \]
    \end{tcolorbox}}
    \pause

    \vspace{6mm}
    Similarly, $\gr^{1}H=F^{1}H/0=F^{1}H$, thus 
    \vspace{3mm}

    \adjustbox{scale=1.2,center}{%
    \begin{tcolorbox}[colback=yellow!25!white,colframe=orange!5!white,text width=8cm]
    \vspace{-3mm}
    \[ \gr^{1}H=\ker\left(d^{0}|_{F^{1}C}\right)[0]\op \left(\frac{F^{1}C^{1}+\im(d^{0})}{\im(d^{0})}\right)[-1] \]
    \end{tcolorbox}}
\end{frame}

\begin{frame}
    \frametitle{\textsc{Baby Example}: define $E_{0}^{p,q}:=(\gr^{p}C)^{p+q}$ and visualize in $\Z^{2}$}
    \begin{columns}
    \hspace{-5mm}
    \begin{column}{0.5\textwidth}
    \begin{tikzpicture}[ampersand replacement=\&]
	\matrix (m) [matrix of math nodes,
	nodes in empty cells, nodes={minimum width=5ex,
	minimum height=5ex, outer sep=-5pt},
	column sep=2ex, row sep=1ex]{
	      \&   \&   \&  \&   \& \\
	    2 \& \color{gray!50}0 \& \color{gray!50}0 \& \color{gray!50}0 \& \color{gray!50}0 \& \\
	    1 \& \color{gray!50}0 \& C^{1}/F^{1}C^{1} \& \color{gray!50}0 \& \color{gray!50}0 \& \\
	    0 \& \color{gray!50}0 \& C^{0}/F^{1}C^{0} \& F^{1}C^{1} \& \color{gray!50}0 \& \\
	    -1\hspace{2mm} \& \color{gray!50}0 \& \color{gray!50}0 \& F^{1}C^{0} \& \color{gray!50}0 \& \\
	    -2\hspace{2mm} \& \color{gray!50}0 \& \color{gray!50}0 \& \color{gray!50}0 \& \color{gray!50}0 \& \\
	    \quad\strut \& -1 \& 0 \& 1 \& 2 \& \strut \\};
	\draw[-stealth] (m-4-3.north) -- node [left] {$d^{0,0}_{0}$}(m-3-3.south);
	\draw[-stealth] (m-5-4.north) -- node [right] {$d^{1,-1}_{0}$}(m-4-4.south);
	\draw[-stealth, ultra thick] (m-7-1.east) -- (m-1-1.east) node[above]{$q$};
	\draw[-stealth, ultra thick] (m-7-1.north) -- (m-7-6.north) node[right]{$p$};
	\draw[gray!50, -stealth] (m-3-2.north) -- (m-2-2.south);
	\draw[gray!50, -stealth] (m-4-2.north) -- (m-3-2.south);
	\draw[gray!50, -stealth] (m-5-2.north) -- (m-4-2.south);
	\draw[gray!50, -stealth] (m-6-2.north) -- (m-5-2.south);
	\draw[gray!50, -stealth] (m-3-3.north) -- (m-2-3.south);
	\draw[gray!50, -stealth] (m-5-3.north) -- (m-4-3.south);
	\draw[gray!50, -stealth] (m-6-3.north) -- (m-5-3.south);
	\draw[gray!50, -stealth] (m-3-4.north) -- (m-2-4.south);
	\draw[gray!50, -stealth] (m-4-4.north) -- (m-3-4.south);
	\draw[gray!50, -stealth] (m-6-4.north) -- (m-5-4.south);
	\draw[gray!50, -stealth] (m-3-5.north) -- (m-2-5.south);
	\draw[gray!50, -stealth] (m-4-5.north) -- (m-3-5.south);
	\draw[gray!50, -stealth] (m-5-5.north) -- (m-4-5.south);
	\draw[gray!50, -stealth] (m-6-5.north) -- (m-5-5.south);
    \end{tikzpicture}
    \pause
    \end{column}
	\hspace{5mm}
    \begin{column}{0.52\textwidth}
	\begin{itemize}
		\vspace{-20mm}
	    \item Since $F^{1}C\subseteq C$ is a subcomplex, $d^{0}\colon C^{0}\to C^{1}$ induces the differential $d^{0,0}_{0}$.
		\vspace{2mm}
		\pause

	    \item For the same reason, $d^{0}|_{F^{1}C}$ has image in $F^{1}C^{1}$, so it induces the differential $d_{0}^{1,-1}$.
		\vspace{2mm}
		\pause

	    \item In general, since $F^{p+1}C\subseteq F^{p}C$ is a subcomplex, the original differentials induce the vertical differentials.
	\end{itemize}
    \end{column}
    \end{columns}
\end{frame}

\begin{frame}
    \frametitle{\textsc{Baby Example}: compute cohomology of the columns $\gr^{p}C$}
    From the $p=0$ column in the ``$E_{0}$-page'' we compute
    \vspace{3mm}

    \adjustbox{scale=1.1,center}{%
    \begin{tcolorbox}[colback=yellow!25!white,colframe=orange!5!white,text width=10cm]
    \vspace{-3mm}
	\[ H(\gr^{0}C)=\left(\frac{\ker(d^{0})+(d^{0})^{-1}(F^{1}C^{1})}{F^{1}C^{0}}\right)[0]\op \left(\frac{C^{1}}{F^{1}C^{1}+\im(d^{0})}\right)[-1] \]
    \end{tcolorbox}}
    \pause

    \vspace{6mm}
    And from the $p=1$ column in the ``$E_{0}$-page'' we compute
    \vspace{3mm}

    \adjustbox{scale=1.1,center}{%
    \begin{tcolorbox}[colback=yellow!25!white,colframe=orange!5!white,text width=10cm]
	\[ H(\gr^{1}C)=\ker\left(d^{0}|_{F^{1}C}\right)[0]\op \left(\frac{F^{1}C^{1}}{\im(d^{0}|_{F^{1}C})}\right)[-1] \]
    \end{tcolorbox}}
\end{frame}

\begin{frame}
    \frametitle{\textsc{Baby Example}: $1^{\mathrm{st}}$ approximation to the $\gr^{0}H$ part}
    We compare $\gr^{0}H$ (left column) to $H(\gr^{0}C)$ (right column):
    \vspace{3mm}

    \adjustbox{scale=1.3,center}{%
    \begin{tcolorbox}[colback=red!15!white,colframe=orange!5!white,text width=8cm]
	\begin{tikzcd}[ampersand replacement=\&]
	    \left(\frac{\ker(d^{0})}{\ker(d^{0}|_{F^{1}C})}\right)[0] \&[-15pt] \&[-15pt] \left(\frac{\ker(d^{0})+(d^{0})^{-1}(F^{1}C^{1})}{F^{1}C^{0}}\right)[0] \\[-20pt]
	    \op \& \& \op \\[-20pt]
	    \left(\frac{C^{1}}{F^{1}C^{1}+\im(d^{0})}\right)[-1] \& \& \left(\frac{C^{1}}{F^{1}C^{1}+\im(d^{0})}\right)[-1] 
	\end{tikzcd}
    \end{tcolorbox}}
    \pause

    \vspace{5mm}
    \begin{itemize}
	\item $H^{1}(\gr^{0}C)$ does compute $\gr^{0}H^{1}$ (bottom row),
	    \pause
	
	    \vspace{3mm}
	\item but $H^{0}(\gr^{0}C)$ is not isomorphic to $\gr^{0}H^{0}$ (top row).
    \end{itemize}
\end{frame}

\begin{frame}
    \frametitle{\textsc{Baby Example}: $1^{\mathrm{st}}$ approximation to the $\gr^{1}H$ part}
    We compare now $\gr^{1}H$ (left column) to $H(\gr^{1}C)$ (right column):
    \vspace{3mm}

    \adjustbox{scale=1.3,center}{%
    \begin{tcolorbox}[colback=red!15!white,colframe=orange!5!white,text width=8cm]
	\begin{tikzcd}[ampersand replacement=\&]
	    \ker\left(d^{0}|_{F^{1}C}\right)[0] \&[-15pt] \&[-15pt] \ker\left(d^{0}|_{F^{1}C}\right)[0] \\[-20pt]
	    \op \& \& \op \\[-20pt]
	    \left(\frac{F^{1}C^{1}+\im(d^{0})}{\im(d^{0})}\right)[-1] \& \& \left(\frac{F^{1}C^{1}}{\im\left(d^{0}|_{F^{1}C}\right)}\right)[-1] 
	\end{tikzcd}
    \end{tcolorbox}}
    \pause

    \vspace{5mm}
    \begin{itemize}
	\item In this case $H^{0}(\gr^{1}C)$ does compute $\gr^{1}H^{0}$ (top row),
	    \pause
	
	    \vspace{3mm}
	\item but $H^{1}(\gr^{1}C)$ is not isomorphic to $\gr^{1}H^{1}$ (bottom row).
    \end{itemize}
\end{frame}

\begin{frame}
    \frametitle{\textsc{Baby Example}: define $E_{1}^{p,q}:=H^{p+q}(\gr^{p}C)$ and plot in $\Z^{2}$}
    \begin{tikzpicture}[ampersand replacement=\&]
	\matrix (m) [matrix of math nodes,
	nodes in empty cells, nodes={minimum width=5ex,
	minimum height=4ex, outer sep=-5pt},
	column sep=8ex, row sep=1ex]{
	    \&  \&  \& \\
	    1 \& \frac{C^{1}}{F^{1}C^{1}+\im(d^{0})} \& \color{gray!50}0 \& \\
	    0 \& \frac{\ker(d^{0})+(d^{0})^{-1}(F^{1}C^{1})}{F^{1}C^{0}} \& \frac{F^{1}C^{1}}{\im\left(d^{0}|_{F^{1}C}\right)} \& \\
	    -1\hspace{2mm} \& \color{gray!50}0 \& \ker\left(d^{0}|_{F^{1}C}\right) \& \\
	    \quad\strut \& 0 \& 1 \& \strut \\};
	\draw[-stealth, ultra thick] (m-5-1.east) -- (m-1-1.east) node[above]{$q$};
	\draw[-stealth, ultra thick] (m-5-1.north) -- (m-5-4.north) node[right]{$p$};
	\draw[-stealth, shorten >=5mm,shorten <=5mm] (m-3-2) -- node [above] {$d^{0,0}_{1}$} (m-3-3.173);
	\draw[shorten >=8mm,shorten <=8mm,gray!50,-stealth] (m-2-2) -- (m-2-3.184);
	\draw[shorten >=4mm,shorten <=8mm,gray!50,-stealth] (m-4-2) -- (m-4-3.179);
    \end{tikzpicture}
    \pause

    \begin{itemize}
	\item The original differential $d^{0}$ induces the differential $d^{0,0}_{1}$.
	    \pause

	\item More generally, if we had started from a longer complex, the original differentials would induce the horizontal differentials.
    \end{itemize}
\end{frame}

\begin{frame}
    \frametitle{\textsc{Baby Example}: $2^{\mathrm{nd}}$ approximation to $\gr H$}
    \begin{itemize}
	\item We obtained the $1^{\mathrm{st}}$ approximation by taking cohomologies on the ``$E_{0}$-page''.
	    \pause 
	    
	\item These $1^{\mathrm{st}}$ approximations were then organized into the ``$E_{1}$-page'' in such a way that $\gr^{p}H^{n}$ would correspond to $E_{1}^{p,n-p}$.
	    \pause
	    
	\item For example, we have seen that
	    \begin{itemize}
		\item $\gr^{0}H^{1}=E_{1}^{0,1}$, altough $\gr^{0}H^{0}\neq E_{1}^{0,0}$;
		    \pause
		\item $\gr^{1}H^{0}=E_{1}^{1,-1}$, although $\gr^{1}H^{1}\neq E_{1}^{1,0}$.
	    \end{itemize}
	    \pause
	\item Continuing with this way to arrange things, we define the ``$E_{2}$-page'' by taking cohomologies at each point of the ``$E_{1}$-page'', that is
	    \[ E_{2}^{p,q}:=H^{p}(E_{1}^{\bullet,q}), \]
	    so that $\gr^{p}H^{n}$ would again correspond to $E_{2}^{p,n-p}$.
    \end{itemize}
\end{frame}

\begin{frame}
    \frametitle{\textsc{Baby Example}: $2^{\mathrm{nd}}$ approximation to the $\gr^{0}H$ part}
    We compare then $\gr^{0}H$ (left column) to $E_{2}^{0,\bullet}$ (right column):
    \vspace{3mm}

    \adjustbox{scale=1.3,center}{%
    \begin{tcolorbox}[colback=green!15!white,colframe=orange!5!white,text width=8cm]
	\begin{tikzcd}[ampersand replacement=\&]
	    \left(\frac{\ker(d^{0})}{\ker(d^{0}|_{F^{1}C})}\right)[0] \&[-15pt] \&[-15pt] \left(\frac{\ker(d^{0})+F^{1}C^{0}}{F^{1}C^{0}}\right)[0] \\[-20pt]
	    \op \& \& \op \\[-20pt]
	    \left(\frac{C^{1}}{F^{1}C^{1}+\im(d^{0})}\right)[-1] \& \& \left(\frac{C^{1}}{F^{1}C^{1}+\im(d^{0})}\right)[-1] 
	\end{tikzcd}
    \end{tcolorbox}}
    \pause

    \vspace{5mm}
    After applying an $\cong$-theorem we see that the two agree!
\end{frame}	

\begin{frame}
    \frametitle{\textsc{Baby Example}: $2^{\mathrm{nd}}$ approximation to the $\gr^{1}H$ part}
    We compare now $\gr^{1}H$ (left column) to $E_{2}^{1,\bullet-1}$ (right column):
    \vspace{3mm}

    \adjustbox{scale=1.3,center}{%
    \begin{tcolorbox}[colback=green!15!white,colframe=orange!5!white,text width=8cm]
	\begin{tikzcd}[ampersand replacement=\&]
	    \ker\left(d^{0}|_{F^{1}C}\right)[0] \&[-15pt] \&[-15pt] \ker\left(d^{0}|_{F^{1}C}\right)[0] \\[-20pt]
	    \op \& \& \op \\[-20pt]
	    \left(\frac{F^{1}C^{1}+\im(d^{0})}{\im(d^{0})}\right)[-1] \& \& \left(\frac{F^{1}C^{1}}{\im(d^{0})\cap F^{1}C^{1}}\right)[-1] 
	\end{tikzcd}
    \end{tcolorbox}}
    \pause

    \vspace{5mm}
    After applying an $\cong$-theorem we see that the two sides agree again!
\end{frame}

\begin{frame}
    \frametitle{\textsc{Baby Example}: reading off the result from the ``$E_{2}$-page''}
    \begin{tikzpicture}[ampersand replacement=\&]
	\matrix (m) [matrix of math nodes,
	nodes in empty cells, nodes={minimum width=5ex,
	minimum height=5ex, outer sep=-5pt},
	column sep=8ex, row sep=1ex]{
	    \&  \&  \& \\
	    1 \& \frac{C^{1}}{F^{1}C^{1}+\im(d^{0})} \& 0 \& \\
	    0 \& \frac{\ker(d^{0})}{\ker\left(d^{0}|_{F^{1}C}\right)} \& \frac{F^{1}C^{1}+\im(d^{0})}{\im(d^{0})} \& \\
	    -1\hspace{2mm} \& 0 \& \ker\left(d^{0}|_{F^{1}C}\right) \& \\
	    \quad\strut \& 0 \& 1 \& \strut \\};
	\draw[-stealth, ultra thick] (m-5-1.east) -- (m-1-1.east) node[above]{$q$};
	\draw[-stealth, ultra thick] (m-5-1.north) -- (m-5-4.north) node[right]{$p$};
    \end{tikzpicture}
    \pause

    \begin{columns}
    \begin{column}{0.5\textwidth}
    \begin{itemize}
	\item $\gr^{0}H\cong E_{2}^{0,0}[0]\op E_{2}^{0,1}[-1]$;
	    \pause

	\item $\gr^{1}H\cong E_{2}^{1,-1}[0]\op E_{2}^{1,0}[-1]$.
	    \pause
    \end{itemize}
    \end{column}
    \begin{column}{0.5\textwidth}
	\[
	\hspace{-3mm}
	    \begin{cases}
		\usebeamercolor[fg]{frametitle}{H^{0}}\color{black}\cong \gr^{0}H^{0}\op \gr^{1}H^{0}\cong \usebeamercolor[fg]{frametitle}{E_{2}^{0,0}\op E_{2}^{1,-1}} \\
		\usebeamercolor[fg]{frametitle}{H^{1}}\color{black}\cong \gr^{1}H^{0}\op \gr^{1}H^{1}\cong \usebeamercolor[fg]{frametitle}{E_{2}^{0,1}\op E_{2}^{1,0}}
	    \end{cases}
	\]
    \end{column}
    \end{columns}
\end{frame}

\begin{frame}
    \frametitle{References}
    \bibliographystyle{alpha}
    \bibliography{main.bib}
\end{frame}
 
\end{document}
