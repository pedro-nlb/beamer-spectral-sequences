\documentclass[notheorems, hyperref={backref}]{beamer}

%% KEY LINES IN THIS TEX FILE %% (enter line number+gg to go)

%
% LOCAL FONT DEFINITIONS -- need to come first
%
%\usepackage{mathpazo}
\usepackage{libertine}
\usepackage[libertine]{newtxmath}
\usefonttheme[onlymath]{serif}

%
% STANDARD PREAMBLE
%
\input{preamble}
\allowdisplaybreaks

%
% ABOUT FONT DEFINTIONS IN THE PREAMBLE
%
% Mathscr for sheaves use \sA, where A can be any letter. Exceptions and additions:
% % \E (vector bundles)
% % \F (coherent sheaves)
% % \G (coherent sheaves)
% % \hom (sheaf hom)
% % \I (ideal sheaves)
% % \L (line bundles)
% % \M (line bundles)
% % \O (structure sheaf)
% % \w (canonical sheaf)
%
% Mathcal use \calA. Exceptions and additions:
% % \U (open cover)
% % \X (families of varieties)
% % \Y (families of varieties)
%
% Mathbb use \bbA. Exceptions and additions:
% % \A (affine space)
% % \C (complex numbers)
% % \Gm (puctured affine line)
% % \k (field)
% % \N (natural numbers)
% % \P (projective space)
% % \Q (rational numbers)
% % \R (real numbers)
% % \V (geometric vector bundle)
% % \Z (integers)
%
% Boldfont for categories use \bfA. Additions:
% % \Cat (categories)
% % \Coh (coherent sheaves)
% % \D (derived category)
% % \Db (bounded derived category)
% % \K (homotopy category)
% % \Mod (modules)
% % \PSh (presheaves)
% % \QCoh (quasi-coherent sheaves)
% % \Set (sets)
% % \Sh (sheaves)
% % \Top (topological spaces)
% % \Vec (vector bundles)
%
% Mathfrak for ideals
% % From \a to \e
% % \m and \n for maximal ideals

%
% THEOREM ENVIRONMENTS
%
% Theorems, propositions, etc (dark green)
\theoremstyle{darkgreentheorem}
\newtheorem{thm}{Theorem}
\newtheorem{lm}[thm]{Lemma}
\newtheorem{prop}[thm]{Proposition}
\newtheorem{cor}[thm]{Corollary}
\newtheorem{conj}[thm]{Conjecture}
% Definitions (dark blue)
\theoremstyle{darkbluedefinition}
\newtheorem{defn}[thm]{Definition}
% Examples (dark red)
\theoremstyle{darkredexample}
\newtheorem{exa}[thm]{Example}
% Remarks (black)
\theoremstyle{remark}
\newtheorem{rem}[thm]{Remark}
\newtheorem{nota}[thm]{Notation}
\newtheorem{fact}[thm]{Fact}
\newtheorem{q}[thm]{Question}
\newtheorem{pbl}[thm]{Problem}

%
% THEOREM CROSS-REFERENCING
%
\crefname{thm}{theorem}{theorems}
\Crefname{thm}{Theorem}{Theorems}
\crefname{lm}{lemma}{lemmas}
\Crefname{lm}{Lemma}{Lemmas}
\crefname{prop}{proposition}{propositions}
\Crefname{prop}{Proposition}{Propositions}
\crefname{cor}{corollary}{corollaries}
\Crefname{cor}{Corollary}{Corollaries}
\crefname{conj}{conjecture}{conjectures}
\Crefname{conj}{Conjecture}{Conjectures}
\crefname{defn}{definition}{definitions}
\Crefname{defn}{Definition}{Definitions}
\crefname{exa}{example}{examples}
\Crefname{exa}{Example}{Examples}
\crefname{rem}{remark}{remarks}
\Crefname{rem}{Remark}{Remarks}
\crefname{nota}{notation}{notations}
\Crefname{nota}{Notation}{Notations}
\crefname{fact}{fact}{facts}
\Crefname{fact}{Fact}{Facts}
\crefname{q}{question}{questions}
\Crefname{q}{Question}{Questions}
\crefname{pbl}{problem}{problems}
\Crefname{pbl}{Problem}{Problems}

%
% MATH OPERATORS
%
\DeclareMathOperator{\Hom}{Hom}
\DeclareMathOperator{\im}{im}

%
% OTHER COMMANDS
%
\newcommand{\ot}{\otimes}
\newcommand{\op}{\oplus}
\newcommand{\filt}{F^{\bullet}}

%
% TITLE PAGE INFORMATION
%
\title[Basic Notions --- Spectral Sequences]{Basic Notions --- Spectral Sequences}
\author{Pedro Núñez}
\institute{University of Freiburg}
\date{10th December 2020}
 
%
% LINKS AND PDF OPTIONS
%
\makeatletter
\hypersetup{
  %pdfauthor={\authors},
  pdftitle={\@title},
  %pdfsubject={\@subjclass},
  %pdfkeywords={\@keywords},
  %pdfstartview={Fit},
  %pdfpagelayout={TwoColumnRight},
  %pdfpagemode={UseOutlines},
  bookmarks,
  colorlinks,
  linkcolor=linkblue,
  citecolor=linkblue,
  urlcolor=linkblue}
\makeatother
\usecolortheme{orchid}
 
\begin{document}
 
\frame{\titlepage}

\begin{frame}
    \frametitle{``You could have invented spectral sequences''}
    \begin{itemize}
	\item This is the title of the expository article \cite{cho06}.
	    \pause
	\item The author tries to convince the reader that they could have come up with the definition of \textit{spectral sequence} themselves.
	    \pause
	\item Our modester goal will be to go through an easy example in detail and hopefully convince ourselves by the end of the talk that the definition of a spectral sequence makes sense.
	    \pause
	\item \textbf{Example:} let $k$ be a field and consider a cochain complex $C^{\bullet}$ of finite dimensional $k$-vector spaces
	    \[ \cdots \to 0\to C^{0}\xrightarrow{d^{0}} C^{1}\xrightarrow{d^{1}} C^{2}\to 0 \to \cdots \]
	    \pause
	    \vspace{-5mm}
	\item \textbf{Goal:} compute the cohomology $H^{\bullet}$ of the complex $C^{\bullet}$
	    \[ H^{\bullet}:=H^{0}(C^{\bullet})\op H^{1}(C^{\bullet})\op H^{2}(C^{\bullet}). \]
    \end{itemize}
\end{frame}

\begin{frame}
    \frametitle{Starting point}
    \begin{itemize}
	\item If we do not have any extra information about $C^{\bullet}$, there is not much more we can do besides applying the definition
	    \[ H^{\bullet}=\ker(d^{0})\op\frac{\ker(d^{1})}{\im(d^{0})}\op \frac{C^{2}}{\im(d^{1})}. \]
	    \pause
	    \vspace{-5mm}
	\item Luckily, $C^{\bullet}$ has often a natural filtration, for example
	    \begin{center}
		\begin{tikzcd}[ampersand replacement=\&]
		    C^{\bullet}=F^{0}C^{\bullet} \& C^{0}\arrow{r}{d^{0}} \&[-10pt] C^{1}\arrow{r}{d^{1}} \&[-10pt] C^{2} \\[-10pt]
		    F^{1}C^{\bullet}\arrow[draw=none]{u}[sloped,auto=false]{\subseteq} \& F^{1}C^{0}\arrow[draw=none]{u}[sloped,auto=false]{\subseteq}\arrow{r} \& F^{1}C^{1}\arrow[draw=none]{u}[sloped,auto=false]{\subseteq}\arrow{r} \& F^{1}C^{2}\arrow[draw=none]{u}[sloped,auto=false]{\subseteq} \\[-10pt]
		    F^{2}C^{\bullet}\arrow[draw=none]{u}[sloped,auto=false]{\subseteq} \& F^{2}C^{0}\arrow[draw=none]{u}[sloped,auto=false]{\subseteq}\arrow{r} \& F^{2}C^{1}\arrow[draw=none]{u}[sloped,auto=false]{\subseteq}\arrow{r} \& F^{2}C^{2}\arrow[draw=none]{u}[sloped,auto=false]{\subseteq} \\[-10pt]
		    0=F^{3}C^{\bullet}\arrow[draw=none]{u}[sloped,auto=false]{\subseteq} \& 0\arrow[draw=none]{u}[sloped,auto=false]{\subseteq}\arrow{r} \& 0\arrow[draw=none]{u}[sloped,auto=false]{\subseteq}\arrow{r} \& 0\arrow[draw=none]{u}[sloped,auto=false]{\subseteq}
		\end{tikzcd}
	    \end{center}
    \end{itemize}
\end{frame}

\begin{frame}
    \frametitle{References}
    \bibliographystyle{alpha}
    \bibliography{main.bib}
\end{frame}
 
\end{document}
